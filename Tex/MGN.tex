\chapter{基于多粒度切分的局部对齐网络}

\section{引言}
近年来,局部信息在计算机视觉的各种任务中越来越经常被用到,这些任务包括但不限于:细粒度分类\cite{wang2018learning}、行人重识别\cite{sun2018beyond}、视觉问答、图像检索等
做好服装检索这个任务的关键就是尽可能的去优化图像的局部信息表示,而学习局部特征最为重要的问题就是局部区域的定位与对齐。以对局部特征不同的学习方式为区分,
现阶段一些比较好的方法可以分为两种路线:基于关键点\cite{wei2016convolutional}和基于语义信息\cite{wei2017glad}。基于关键点的做法需要训练数据具有关键点的标注,
而直接基于语义信息去做则不需要额外的标注。

在第三章中提出的基于注意力机制的局部对齐网络就是用语义信息去做局部的定位,基于注意力的做法比传统的对图像按照一些规则切分更加准确和灵活,在本章,我们介绍一种
具有创新性的切分方式去学习局部特征。

\section{方法}
传统的基于切分去学习局部特征的做法大多数是对原图做切分\cite{li2017learning}

