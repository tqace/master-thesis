\chapter{总结与展望}
近年来,随着电子商务领域的快速发展和人们消费水平的提高,网购逐渐成为人们消费的最大途径,为了提高消费者的购物体验,对同款服装检索算法的研究与设计愈发具有重大意义。
传统的服装检索算法主要基于文本(TBIR),而这种方式越来越不能满足人们逐渐精细的检索需求,随着深度学习的爆发式发展,基于内容的同款服装检索算法成为了主流。
现阶段,对局部区域的特征提取和对齐是图像检索的一个热门研究方向,基于此,本文提出两种基于深度学习的局部对齐网络。

提出基于注意力机制的局部对齐网络。深度学习中的注意力机制借鉴自人类的视觉信息处理机制,旨在将更多的资源分配在更加关键的区域,所提出的方法设计了多路并行的注意力模块,
以学习不同的局部注意力分布,每个分支的注意力模块都包括空间注意力和通道注意力两个子模块。通过可视化分析,我们发现这种网络设计形式虽然有多个相同结构的分支,
但是不同的分支可以提取不同的局部区域信息。在此基础上,基于网络采用的在线三元组损失(Online Triplet Loss),我们提出一种跨域样本挖掘的可迭代算法,该算法可以基于
源域训练的模型在目标域挖掘无标签的样本,并生成伪标签用于下一轮次的训练。这种算法可有效提升模型对目标域的泛化能力,且满足增量学习的特性,适合企业的实际应用场景。

提出基于多粒度切分的局部对齐网络。本文实验发现,有别于对原图进行局部区域的裁减,对特征图进行切分以提取局部特征可以得到很好的效果。该方法有效结合了全局和局部特征,
局部分支采用多任务学习的方式;局部分支以三种不同的切分方式(横向、纵向和环形)分为三个分支,可以更加有效的保留局部信息的完整性。借鉴GoogLnet的多尺度卷积,以及FPN的
特征金字塔思想,提出多粒度的切分方式,融合多种局部感受野的信息,有效增强特征表达能力。此外,得益于最大池化操作对图像平移、缩放的抗干扰能力,结合平均池化和最大池化,
最大限度的提取局部区域的语义信息。

上述两种算法均是基于局部信息的对齐网络,基于注意力机制的网络有着局部区域的自适应对齐的特性;而基于切分的局部对齐网络则依赖于多样的切分方式,使用不同的切分策略可以
提升关键局部信息对齐的可能性,对旋转或者平移等情况有着有效的泛化能力。本课题在实际应用时采用模型融合的方法,两种算法分别训练得到对应的深度模型,模型大小均在300M以内,在两块Titan Xp显卡上测试算法的前传时间在0.5s以内。此外,两个模型
所提取特征的拼接之后作为输入图像的最终特征表示,可以进一步提升检索精度,在上装、下装和群装的性能可以达到86\%、81.2\%、74.8\%,这说明两种算法有着较强的互补性。

如何更好的提取和对齐输入图像的局部区域仍然是今后同款服装检索的重要研究方向,在本文的工作的基础上,依然有可以改进的细节。对于基于注意力机制的局部对齐网络,每个
分支采取独立训练的方式,为了使不同分支的局部注意力更加多样化,可以考虑在不同分支的输出特征之间引入度量学习进行监督,以强迫不同分支所提取的局部特征具有差异性。
另外,如何有效结合软注意力和硬注意力机制也是一个十分值得探究的方向。
